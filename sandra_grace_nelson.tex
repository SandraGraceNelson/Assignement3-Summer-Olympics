\documentclass{article}
\usepackage[utf8]{inputenc}
\usepackage{amsmath}
\usepackage{enumitem}
\usepackage{graphicx}
\title{\textbf{A Graph Theory-Driven Approach for Optimizing Chemical Reaction Prediction with Machine Learning}}
\author{Sandra Grace Nelson}
\begin{document}
\maketitle

\section{Problem Statement}
The accurate prediction of chemical reactions is a crucial aspect in the field of chemistry. The current traditional methods for predicting chemical reactions are often slow and lack precision, leading to inefficiencies in the industry. However, for this purpose, machine learning  models are often trained solely on real-world data, which can introduce biases and limitations that are difficult to identify and can negatively impact the accuracy of the predictions. Additionally, collecting high-quality and diverse data for training these models is time-consuming and presents a significant challenge. To address these issues, developing a more robust and reliable approach to predicting chemical reactions using machine learning techniques that incorporates diverse data sources and minimizes the potential for biases and limitations is essential. Also, it is important to study the effects of biased data on the learning ability of machine learning models. 

\section{Objectives}
To mitigate the issues stated in the problem statement, this project proposes to create a machine learning model that predicts chemical reactions using molecular graphs. This project's main objective is to study machine learning models' learning ability on biased data. The model will be trained on synthetic data that mimics chemical reactions between molecules to deal with the limitations and biases present in real-world data, leading to more robust and accurate predictions. Aims of this project include,
\begin{itemize}
    \item  To study the learning ability of machine learning models on biased data.
    \item To model the chemical reaction prediction problem as a graph theory problem.
    \item To create synthetic data that represents chemical reactions using the Python NetworkX library.
    \item To train a machine learning model on the synthetic data to predict the chemical reaction.
\end{itemize}

\section{Proposed Methodology}

\subsection{Synthetic Data Generation}
As mentioned above, this project models the chemical reaction prediction problem as a graph theory problem. So the first step is to represent molecules as networks, where the nodes in the network correspond to the atoms, and the edges represent the bonds between the atoms. Clearly, the degree of each node will be the valency of each atom. The obtained graph is a multigraph. The four elements, Carbon, Nitrogen, Hydrogen, and Oxygen, are considered because they are the most common elements found in organic molecules. The products will be created as new instances of these molecules by alternating existing bonds in the network while preserving the valencies of each atom. After creating these instances of the molecules, the VF2 algorithm is used to find the maximum common subgraph (MCS) between the corresponding multigraph of the reactant side and product side. The MCS represents the structure common to the reactant and product networks. Then it will be easy to complete the bijection between the networks to attain isomorphism. The bijection is a one-to-one correspondence between the network nodes, ensuring that the reactants' structure is correctly represented in the model. This will act as the ground truth for our data generation. So the synthetic data for this project will be the graphs of the reactant and product sides in which the bijection is satisfied.

\subsection{Training and Evaluation of Machine Learning Models}
The synthetic data will be used to train different machine learning architectures to predict possible chemical reactions between the molecules. The machine learning model's performance will be evaluated by comparing its predictions with the ground truth synthetic data. The models will be tested for their ability to learn the pattern and predict the chemical reactions correctly. Then the evaluation will determine the best machine learning architecture that accurately predicts chemical reactions for further study. In simple words, our approach is to train the machine when the reaction cycle is a 4-cycle or 6-cycle and try to predict when the reaction cycle is an 8-cycle or 16-cycle etc. Then the goal is to study how far the machine can learn the pattern when we change the cycle length, and we can compare the performance with the actual randomly generated outputs.
Furthermore, to assess the reliability and adaptability of the developed model, subsets of the synthetic data with a diverse range of complexity and bias will be employed in the training phase. This will help uncover the strong and weak points of the model and determine the extent to which it can learn and handle different types of data. Simply, it will help to study the learning ability of machine learning models on biased data.

\section*{Timeline to Complete The Project}
\textbf{\textit{The estimated time to complete this project is approximately 8 months, starting from 15 February 2023 with the following detailed timeline:}}
\begin{itemize}
	\item {\textbf{Month 1: Preliminary Study}}
	\begin{itemize}
		\item Task 1: Chemistry part - Studying element properties, molecular properties, molecular graphs and chemical reactions  
		\item Task 2: Graph theory part - Studying basic concepts in graph theory, including the problem of MCS and its complexity and  hardness and on the class of problems and related algorithms where the MCS is solvable in polynomial time and VF2 algorithm
            \item Task 3: Familiarizing with the NetworkX library in python to create molecular graphs
	\end{itemize}
	\item {\textbf{Month 2-4: Synthetic Data Generation}}
	\begin{itemize}
		\item Task 4: Generating the networks representing molecules using the NetworkX library
  		\item Task 5: Finding the Maximum Common Subgraphs (MCS) using the VF2 algorithm
		\item Task 6: Completing bijection to attain isomorphism
		\item Task 7: Finalizing the data after prepossessing it
	\end{itemize}
	\item {\textbf{Month 5-6: Model Development and Training}}
	\begin{itemize}
		\item Task 8: Training different machine learning models for predicting the chemical reactions
  	\end{itemize}
        \item {\textbf{Month 7: Model Testing}}
        \begin{itemize}
		\item Task 9: Evaluating the models in predicting chemical reactions, deciding which model performs well, and further processing will be done to improve the learning
	\end{itemize}
	\item {\textbf{Month 8: Project Report}}
	\begin{itemize}
		\item Task 10: Project report preparation
	\end{itemize}
\end{itemize}

\end{document}
